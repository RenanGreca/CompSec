% !TEX encoding = UTF-8 Unicode
% This is "sig-alternate.tex" V2.1 April 2013
% This file should be compiled with V2.5 of "sig-alternate.cls" May 2012
%
% This example file demonstrates the use of the 'sig-alternate.cls'
% V2.5 LaTeX2e document class file. It is for those submitting
% articles to ACM Conference Proceedings WHO DO NOT WISH TO
% STRICTLY ADHERE TO THE SIGS (PUBS-BOARD-ENDORSED) STYLE.
% The 'sig-alternate.cls' file will produce a similar-looking,
% albeit, 'tighter' paper resulting in, invariably, fewer pages.
%
% ----------------------------------------------------------------------------------------------------------------
% This .tex file (and associated .cls V2.5) produces:
%       1) The Permission Statement
%       2) The Conference (location) Info information
%       3) The Copyright Line with ACM data
%       4) NO page numbers
%
% as against the acm_proc_article-sp.cls file which
% DOES NOT produce 1) thru' 3) above.
%
% Using 'sig-alternate.cls' you have control, however, from within
% the source .tex file, over both the CopyrightYear
% (defaulted to 200X) and the ACM Copyright Data
% (defaulted to X-XXXXX-XX-X/XX/XX).
% e.g.
% \CopyrightYear{2007} will cause 2007 to appear in the copyright line.
% \crdata{0-12345-67-8/90/12} will cause 0-12345-67-8/90/12 to appear in the copyright line.
%
% ---------------------------------------------------------------------------------------------------------------
% This .tex source is an example which *does* use
% the .bib file (from which the .bbl file % is produced).
% REMEMBER HOWEVER: After having produced the .bbl file,
% and prior to final submission, you *NEED* to 'insert'
% your .bbl file into your source .tex file so as to provide
% ONE 'self-contained' source file.
%
% ================= IF YOU HAVE QUESTIONS =======================
% Questions regarding the SIGS styles, SIGS policies and
% procedures, Conferences etc. should be sent to
% Adrienne Griscti (griscti@acm.org)
%
% Technical questions _only_ to
% Gerald Murray (murray@hq.acm.org)
% ===============================================================
%
% For tracking purposes - this is V2.0 - May 2012

\documentclass{sig-alternate-05-2015}
% Non-unicode must die
\usepackage[utf8]{inputenc}
\usepackage[portuguese]{babel}

\usepackage{alltt}
\usepackage{float}
\usepackage{listings}

\begin{document}

\lstset{language=C,breaklines=true}

\lstset{
  literate={í}{{\'i}}1
           {ç}{{\c c}}1
		   {¬}{{$\neg$}}1
		   {ã}{{\~a}}1
		   {ê}{{\^e}}1
}

\hyphenation{pri-mei-ros in-clu-í-ram in-dis-po-ní-vel ne-ces-sá-ria pa-la-vras}

% Copyright
%\setcopyright{acmcopyright}
%\setcopyright{acmlicensed}
\setcopyright{rightsretained}
%\setcopyright{usgov}
%\setcopyright{usgovmixed}
%\setcopyright{cagov}
%\setcopyright{cagovmixed}


% DOI
%\doi{}

% ISBN
%\isbn{}

%Conference
%\conferenceinfo{PLDI '13}{June 16--19, 2013, Seattle, WA, USA}

%\acmPrice{\$15.00}

%
% --- Author Metadata here ---
%\conferenceinfo{WOODSTOCK}{'97 El Paso, Texas USA}
%\CopyrightYear{2007} % Allows default copyright year (20XX) to be over-ridden - IF NEED BE.
%\crdata{0-12345-67-8/90/01}  % Allows default copyright data (0-89791-88-6/97/05) to be over-ridden - IF NEED BE.
% --- End of Author Metadata ---

\newcommand{\banner}{\textit{banner} }

\floatstyle{ruled}
\newfloat{log}{thp}{lop}
\floatname{log}{Log}

\newfloat{program}{thp}{lop}
\floatname{program}{Programa}

\title{Processo de quebra de senha usando dicionário e força bruta}
%\subtitle{[Extended Abstract]
%\titlenote{A full version of this paper is available as
%\textit{Author's Guide to Preparing ACM SIG Proceedings Using
%\LaTeX$2_\epsilon$\ and BibTeX} at
%\texttt{www.acm.org/eaddress.htm}}}
%
% You need the command \numberofauthors to handle the 'placement
% and alignment' of the authors beneath the title.
%
% For aesthetic reasons, we recommend 'three authors at a time'
% i.e. three 'name/affiliation blocks' be placed beneath the title.
%
% NOTE: You are NOT restricted in how many 'rows' of
% "name/affiliations" may appear. We just ask that you restrict
% the number of 'columns' to three.
%
% Because of the available 'opening page real-estate'
% we ask you to refrain from putting more than six authors
% (two rows with three columns) beneath the article title.
% More than six makes the first-page appear very cluttered indeed.
%
% Use the \alignauthor commands to handle the names
% and affiliations for an 'aesthetic maximum' of six authors.
% Add names, affiliations, addresses for
% the seventh etc. author(s) as the argument for the
% \additionalauthors command.
% These 'additional authors' will be output/set for you
% without further effort on your part as the last section in
% the body of your article BEFORE References or any Appendices.

\numberofauthors{2} %  in this sample file, there are a *total*
% of EIGHT authors. SIX appear on the 'first-page' (for formatting
% reasons) and the remaining two appear in the \additionalauthors section.
%
\author{
% You can go ahead and credit any number of authors here,
% e.g. one 'row of three' or two rows (consisting of one row of three
% and a second row of one, two or three).
%
% The command \alignauthor (no curly braces needed) should
% precede each author name, affiliation/snail-mail address and
% e-mail address. Additionally, tag each line of
% affiliation/address with \affaddr, and tag the
% e-mail address with \email.
%
% 1st. author
\alignauthor
Renan Domingos Merlin Greca\\
			\affaddr{Universidade Federal do Paraná}\\
		    \email{renangreca@gmail.com}
% 2nd. author
\alignauthor
José Robyson Aggio Molinari\\
			\affaddr{Universidade Federal do Paraná}\\
			\email{aggio13@hotmail.com}
}
\date{5 September 2016}
% Just remember to make sure that the TOTAL number of authors
% is the number that will appear on the first page PLUS the
% number that will appear in the \additionalauthors section.

\maketitle

\section{Introdução}
Este é o relatório do segundo trabalho da disciplina de Introdução à Segurança Computacional, cursada na Universidade Federal do Paraná no segundo semestre de 2016.
O trabalho era dividido em três partes, cada uma com um objetivo:
(1) criar um arquivo de dicionário a partir de um diretório contendo arquivos \texttt{txt};
(2) quebrar, utilizando um algoritmo de força bruta, uma senha dado seu hash MD5;
e (3) utilizar informações obtidas nas partes (1) e (2) para quebrar um arquivo de senhas.

Cada parte do trabalho continua seus próprios desafios, que são elaborados adiante.

\section{Desenvolvimento}
Como dito na introdução, este trabalho foi dividido em três partes.
Nesta seção, são descritos detalhes sobre os problemas e as soluções de cada uma delas.

\subsection{Construção do arquivo dicionário}
O primeiro objetivo do trabalho foi desenvolver um programa em C, chamado \texttt{wordlist}, para processar arquivos \texttt{txt} em um diretório e criar um arquivo dicionário.
Para tal, o programa percorre todas as palavras dos arquivos de texto (separadas por espaço ou quebra de linha) e os adiciona a uma lista, evitando palavras duplicadas. A biblioteca \texttt{tinydir} \cite{tinydir} é usada para facilitar a iteração sobre os arquivos no diretório. Ao final da execução, esta lista é escrita em um novo arquivo de texto, chamado \texttt{dict.txt}.

Para compilar o programa, basta executar:
\begin{verbatim}
make wordlist
\end{verbatim}

E, para executá-lo, utiliza-se o seguinte comando:
\begin{verbatim}
wordlist <dir-path>
\end{verbatim}

O argumento \texttt{dir-path} é o caminho do diretório contendo os arquivos de texto.

Os Programas 1 e 2 mostram, respectivamente, os laços para iterar sobre arquivos no diretório e sobre palavras em cada arquivo.
Juntos, eles permitem a construção do arquivo dicionário.

\begin{program}
\begin{lstlisting}
tinydir_open_sorted(&dir, argv[1]);

// Skip the first two files, '.' and '..'.
for (i = 2; i < dir.n_files; i++) {
    tinydir_file file;
    tinydir_readfile_n(&dir, &file, i);
    
    read_words(file.path, words, &count, WORD_MAX);
}
\end{lstlisting}
\caption{Laço para iterar sobre os arquivos de texto.}
\end{program}

\begin{program}
\begin{lstlisting}
while (*count < max) {

    // Read a word from the file
    if (fscanf(f, "%s", temp) != 1) {
        break;
    }

    // If word isn't already in words, add it to the array
    if (!exists(temp, words, (*count))) {
        words[(*count)] = strdup(temp);
        (*count)++;
    }
}
\end{lstlisting}
\caption{Laço para iterar as palavras de cada arquivo.}
\end{program}

\subsection{Quebra de senha por força bruta}
A segunda parte do trabalho foi a mais interessante delas.
Seu objetivo era desenvolver um programa, \texttt{brutexor}, com duas funcionalidaes: primeiro, descobre uma senha com hash conhecido e, segundo, usa essa senha para decifrar uma dica que seria necessária na terceira parte do trabalho.

Para garantir que estávamos no caminho certo, foi utilizada uma \textit{rainbow table} encontrada na Internet para descobrir a senha a partir do hash MD5 e, então, decifrar a dica.
\textit{Rainbow tables} são tabelas pré-computadas de hashes, que tornam a quebra desse tipo de cifra trivial.
A tabela utilizada contém quase um bilhão e meio de palavras — com ela, encontramos facilmente a palavra que desejávamos:

\begin{verbatim}
	MD5:     8d7b356eae43adcd6ad3ee124c3dcf1e
	Palavra: FACIL
\end{verbatim}

Com a chave conhecida, o processo de decifrar a dica era simples: bastou efetuar operações de XOR (ou exclusivo, em C representado pelo operador \texttt{\^}) entre a sequência de caracteres detalhada no enunciado do trabalho e a chave descoberta.
Isso está descrito no Programa 3.
Como a chave já era conhecida, bastou esse algoritmo para decifrar a dica:

\begin{verbatim}
	Cadeia:  0b3430202f27052205292128222619342322272d
	Dica:    MusicaDaLegiaoUrbana
\end{verbatim}

\begin{program}
\begin{lstlisting}
char input[20] = {0x0b, 0x34, 0x30, 0x20, 0x2f, 0x27, 0x05, 0x22, 0x05, 0x29, 0x21, 0x28, 0x22, 0x26, 0x19, 0x34, 0x23, 0x22, 0x27, 0x2d};
char key[5] = "FACIL";
char output[20];

for (i=0; i<20; i++) {
    output[i] = input[i] ^ key[i % strlen(key)];
}
\end{lstlisting}
\caption{O algoritmo que decifra a dica usando a chave.}
\end{program}

A seguir, seria necessário criar o programa para descobrir a chave através de exaustão utilizando o hash MD5 dado.
Para isso, primeiro foi criado um breve algoritmo que calculava o MD5 de ``FACIL'' e o comparava com o hash conhecido, como visto no Programa 4.
Quando compilado no Linux, o programa usa a biblioteca do OpenSSL para calcular o hash, enquanto no macOS é usada a biblioteca CommonCrypto.

\begin{program}
\begin{lstlisting}
void compute_md5(char *str, 
	unsigned char digest[17]) {
    MD5_CTX ctx;
    MD5_Init(&ctx);
    MD5_Update(&ctx, str, strlen(str));
    MD5_Final(digest, &ctx);
    digest[16] = '\0';
}

char key[20] = "FACIL";
unsigned char digest[17];

compute_md5(key, digest);
if (!strcmp((char *) hash,
	(char *) digest)) {
    printf("word found: %s\n", key);
}
\end{lstlisting}
\caption{Algoritmo para testar o hash da chave.}
\end{program}

Por último, foi desenvolvido um algoritmo para percorrer todas as palavras possíveis de 1 a 8 caracteres utilizando um alfabeto alfanumérico (letras maiúsculas, minúsculas e algarismos), calcular o MD5 de cada um e compará-lo ao conhecido.
Várias iterações desse algoritmo foram escritas, até ser encontrada uma versão que encontrasse a senha em tempo aceitável.

Para desenvolver o algoritmo, o problema foi tratado como uma busca em árvore, onde cada caracter é um nó e os possíveis caracteres seguintes são filhos do caracter anterior.
Devido às condições da palavra descritas no parágrafo anterior, o resultado é uma árvore em que cada nó tem 62 filhos ($26\times 2$ letras mais $10$ algarismos) e a profundidade máxima é 8.
O número total de nós é $62^8$ (aproximadamente 218 trilhões), um valor nada trivial para percorrer.
Nenhuma estrutura de dados similar a uma árvore foi desenvolvida; a abstração foi usada apenas para poder aplicar algoritmos conhecidos ao problema.

A versão inicial utilizava uma busca em profundidade.
A primeira palavra testada era ``A'', seguida por ``AA'' até chegar em ``AAAAAAAA'', e então percorrer todas as palavras que começam com sete A's e um oitavo caracter.
Quando todas as palavras começadas em `A' fossem percorridas, ele então partiria para aquelas que começam em `B'.
O problema dessa solução é que percorrer palavras mais longas é muito demorado — o cálculo do MD5 em si não muda, mas o número de palavras possíveis cresce exponencialmente com cada caracter adicionado.
Portanto, se a senha fosse simplesmente a palavra ``0'', o programa levaria dezenas de horas para encontrá-la, pois precisaria percorrer  todas as palavras que iniciam com todas as letras do alfabeto maiúsculas e minúsculas.
Esta função é mostrada no Programa 5.

\begin{program}
\begin{lstlisting}
char *brute_force(char *word, 
           int index, int max_len) {
    int i;
    for (i=0; i<62; i++) {
        word[index] = alphabet[i];
        word[index+1] = '\0';

        compute_md5(word, digest);

        if (!strcmp((char *) hash,
        	(char *) digest)) {
            return word;
        }
        if (index < max_len) {
            char *res = brute_force(word, index+1, max_len);
            if (res != NULL) {
                return res;
            }
        }
    }
    return NULL;
}
\end{lstlisting}
\caption{Algoritmo de busca em profundidade para encontrar a senha.}
\end{program}

A próxima versão do algoritmo incorporou paralelização em CPU utilizando a biblioteca OpenMP.
Dessa forma, cada thread da CPU seria responsável por processar palavras iniciadas com certas letras (um conjunto de letras é atribuído para cada thread).
Apesar do processo acelerado pela computação paralela, a busca em profundidade permaneceu extremamente lenta.
O Programa 6 mostra o laço e os comandos OpenMP para permitir a paralelização.
A variável \texttt{thread\_alphabet} contém os segmentos do alfabeto que devem ser usados como caracteres iniciais para cada thread.
A thread 0, por exemplo, é responsável por testar todas as palavras que iniciam em `A' e outras letras de acordo com o número de threads. 

\begin{program}
\begin{lstlisting}
#pragma omp parallel
{
    #pragma omp single
    for (i=0; i<8; i++) {

        #pragma omp task
        {
            int tid = omp_get_thread_num();

            int j;
            for (j=0; j<strlen(thread_alphabet[tid]) && !done; j++) {
                char *word = malloc(9 * sizeof(char));
                sprintf(word, "%c", thread_alphabet[tid][j]);
                brute_force(word, 
                    1, MAX_LEN);                        
            }
        }
    }
    #pragma omp taskwait
}\end{lstlisting}
\caption{Comandos OpenMP e laço para paralelizar execução.}
\end{program}

\begin{program}
\begin{lstlisting}
int k;
for (k=1; (k<=MAX_LEN) && !done; k++) {
    printf("thread %d is testing words of length %d\n", tid, k);
    int j;
    for (j=0; j<thread_alphabet_len && !done; j++) {
        char *word = malloc(9 * sizeof(char));
        sprintf(word, "%c", thread_alphabet[tid][j]);
        bruteForce(word, 1, k);                        
    }
}
\end{lstlisting}
\caption{Laço para ``forçar'' a busca em largura.}
\end{program}

Portanto, foi cogitado usar busca em largura ao invés de profundidade.
O algoritmo tradicional de busca em largura exige uma fila FIFO (first in, first out) para armazenar as próximas palavras que devem ser testadas.
Quando a palavra ``A'' é testada, o algoritmo coloca todas as palavras com dois caracteres que começam com ``A'' na fila.
Isso permite que todas as palavras curtas sejam testadas antes das palavras mais compridas e, caso a palavra tenha menos que oito caracteres, a execução será notavelmente mais rápida.
No entanto, a fila FIFO ocupa espaço demasiado em memória — especialmente quando o algoritmo está paralelizado em várias threads.
Nos nossos testes, em um computador com 8GB de memória principal, a RAM era exaustada durante os testes de palavras com três caracteres, tornando essa versão do algoritmo inviável.

A solução final foi desenvolver um processo que ``forçava'' o algoritmo de busca em profundidade, que utiliza memória desprezível, a testar as palavras em ordem de tamanho.
Para tal, há um laço adicional na hora de chamar a função de força bruta em cada thread.
Esse laço faz com que a função seja chamada oito vezes, com o tamanho máximo das palavras aumentado a cada iteração.
Assim, na primeira iteração a função testará apenas palavras com um caracter, até que na última sejam testadas apenas palavras com oito.
Esse laço adicional é mostrado no Programa 7.
Vale observar que a variável \texttt{done} é usada para que todas as threads parem a execução quando uma delas encontrar a senha.

O Programa 8 é a versão final da função de força-bruta, incluindo todas as modificações para a paralelização e a busca em largura.
Os condicionais também foram reposicionados para otimizar a execução.
A variável global \texttt{key} armazena a senha encontrada antes de encerrar o procedimento.

\begin{program}
\begin{lstlisting}
void brute_force(char *word,
         int index, int max_len) {

    if (done) return;
    
    if (index == max_len) {
    	// Bottom of recursion, test word
        unsigned char digest[16];
        compute_md5(word, digest);

        if (!strcmp((char *) hash, 
        		(char *) digest)) {
        	// Word found, end execution
            done = true;
            key = strdup(word);
        }
    } else if (index < max_len) {
    	// Word not long enough, go further
        int i;
        for (i=0; i<alphabet_len && !done; i++) {
            word[index] = alphabet[i];
            word[index+1] = '\0';

            bruteForce(word, index+1, max_len);
        }
    }
}
\end{lstlisting}
\caption{Versão final da função de força bruta}
\end{program}

Por fim, o programa sofreu pequenas alterações para funcionar com ou sem paralelização e independente do número de threads disponíveis no computador.
O Programa 9 mostra o processo para criar blocos do alfabeto que serão alocados para cada thread disponível.
A constante OPENMP é definida apenas se o programa for compilado para ser compatível com multiprocessamento.

\begin{program}
\begin{lstlisting}
// If OpenMP is disabled, number of threads is 1.
#if defined(OPENMP)
    int num_threads = omp_get_num_threads();
#else
    int num_threads = 1;
#endif

char *thread_alphabet[num_threads];
int thread_alphabet_len = ((alphabet_len/num_threads)+1);

// Each thread gets a portion of the alphabet allocated to it. 
for (i = 0; i < num_threads; ++i) {
    thread_alphabet[i] = malloc( thread_alphabet_len * sizeof(char) );
    thread_alphabet[i][0] = '\0';
}

char temp[2];
for (i=0; i<alphabet_len; i++) {
    sprintf(temp, "%c", alphabet[i]);
    strcat(thread_alphabet[i % num_threads], temp);
}\end{lstlisting}
\caption{Alocação de blocos do alfabeto}
\end{program}

\subsection{Quebra de arquivo de senha usando\\ arquivo dicionário}

O objetivo da terceira e última parte do trabalho era usar a dica encontrada na segunda parte para encontrar arquivos de texto e então usar o \texttt{wordlist} para gerar um arquivo dicionário.
Esse arquivo dicionário deveria então ser usado como argumento para o programa John The Ripper, uma solução open-source para quebra de senhas, para quebrar o arquivo \texttt{\_etc\_shadow.txt}.

Dada a dica encontrada na segunda parte, seria necessário criar um diretório contendo arquivos de texto com letras de músicas da banda Legião Urbana.
Portanto, foi desenvolvido um script em Python, \texttt{crawler.py} que utiliza a biblioteca BeautifulSoup para pegar as letras das músicas do site \textbf{\url{letras.mus.br}}.
O script limpa o texto de tags HTML e salva cada letra em um arquivo de texto separado, usando o título de cada canção como nome.
Ao rodar o \texttt{wordlist} com essas letras, o resultado é um arquivo \texttt{dict.txt} com 5927 palavras distintas.

Após a execução do script Python e do \texttt{wordlist}, é possível utilizar esses dados para quebrar a senha com o John The Ripper utilizando o seguinte comando:

\begin{verbatim}
john --wordlist=dict.txt _etc_shadow.txt
\end{verbatim}

Dessa forma, foi possível quebrar uma das senhas presentes no arquivo, a do usuário \textit{eva} — ``Cavalos-marinhos'', palavra presente na letra da canção ``Vento no Litoral''.
Através de conversas com colegas, foi descoberto que a senha do usuário \textit{beto} era ``cavalo-marinho'', mas essa palavra não está presente em nenhuma letra.
Para verificar isso, foi criado um arquivo de texto contendo apenas essas duas palavras; com ele, o John The Ripper foi capaz de quebrar ambas senhas.

\section{Resultados e conclusões}

Para as execuções dos algoritmos, dois computadores foram usados: ``SgtPepper'', um MacBook Pro, e ``Packard'', um desktop customizado.
O programa \texttt{wordlist} roda rapidamente em ambos computadores, então sua análise não é muito interessante.

O segundo programa, \texttt{brutexor}, pode ser extremamente lento ou rápido dependendo de algumas condições, portanto há mais a ser dito sobre ele.
Usando a primeira versão do algoritmo, usando busca em profundidade e sem paralelização, o programa rodou por mais de vinte horas sem sucesso — essa demora motivou o desenvolvimento de versões mais eficientes.

\begin{center}
  \begin{tabular}{ l | l | l }
    \hline
    \textbf{Nome} & ``SgtPepper'' & ``Packard'' \\ \hline \hline
    S.O. & macOS 10.12 & Linux Mint 17 \\ \hline
    CPU & Intel Core i5-4308U & Intel Core i7-2600 \\ \hline
    Clock base & 2.8GHz & 3.4GHz \\ \hline
    Turbo boost & 3.3GHz & 3.8GHz \\ \hline
    Cache L3 & 3MB & 8MB \\ \hline
    Cores/threads & 2/4 & 4/8 \\ \hline
    RAM & 8GB DDR3L & 8GB DDR3 \\ \hline
    Armz. & SSD 512GB & HDD 320GB \\ \hline
    \hline
  \end{tabular}
\end{center}

O Packard, que dispõe de um processador quad-core com \textit{hyperthreading}, foi capaz de realizar o algoritmo de força bruta em oito threads paralelas.
Devido à forma como as letras foram distribuídas, muito rapidamente a thread nº 5 chega à senha ``FACIL'' (pois ela já começa testando as palavras iniciadas em `F').
Portanto, a execução do programa \texttt{brutexor} nesse computador leva, aproximadamente, meio segundo.

Devido a complicações em instalar a biblioteca OpenMP no macOS, o programa foi testado no SgtPepper sem utilizar paralelização.
Rodando em apenas uma thread, o \texttt{brutexor} levou aproximadamente 38 segundos para encontrar a senha desejada.

Um fator interessante deste problema é que, dependendo da senha a ser encontrada, ele pode executar em décimos de segundos ou dezenas de horas.
Em particular, qualquer senha de oito caracteres levaria muito tempo para ser encontrada — e portanto a busca em largura foi adotada.
A senha começar com `F' também ajuda, pois é uma das primeiras letras do alfabeto e será processada rapidamente por uma das threads.
Sob a presunção que senhas curtas são mais comuns, acreditamos que o caminho adotado é o mais lógico para este tipo de problema.

O tempo de execução do John The Ripper em ambos computadores também foi comparado.
No Packard, encontrar a senha ``Cavalos-marinhos'' (e descobrir que as outras não estavam no dicionário) levou 51 segundos, enquanto, no SgtPepper, o processo levou cerca de 30 segundos.
Isso ocorre porque o Core i5 presente no SgtPepper, apesar de ser mais lento que o Core i7 do Packard, é da família Haswell, que incorporou modificações na arquitetura que favoreceram funções criptográficas, além do armazenamento em SSD que acelera a leitura do arquivo de dicionário. 

Por fim, o desenvolvimento do trabalho nos mostrou que o processo para quebrar senhas com oito ou mais caracteres pode ser incrivelmente demorado, caso não se tenha uma \textit{rainbow table} completa do algoritmo de hash usado; portanto métodos de engenharia social estão cada vez mais comuns entre atacantes.
As fases de otimização do algoritmo de força bruta também foram fascinantes por um ponto de vista computacional, por mostrarem como formas antigas de segurança, que é o caso do MD5, tornam-se obsoletas com o avanço tecnológico dos microprocessadores..

%
% The following two commands are all you need in the
% initial runs of your .tex file to
% produce the bibliography for the citations in your paper.
\bibliographystyle{abbrv}
\bibliography{sigproc}  % sigproc.bib is the name of the Bibliography in this case
% You must have a proper ".bib" file
%  and remember to run:
% latex bibtex latex latex
% to resolve all references
%
% ACM needs 'a single self-contained file'!
%
%APPENDICES are optional
%\balancecolumns
%\balancecolumns % GM June 2007
% That's all folks!
\end{document}
