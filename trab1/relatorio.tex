% !TEX encoding = UTF-8 Unicode
% This is "sig-alternate.tex" V2.1 April 2013
% This file should be compiled with V2.5 of "sig-alternate.cls" May 2012
%
% This example file demonstrates the use of the 'sig-alternate.cls'
% V2.5 LaTeX2e document class file. It is for those submitting
% articles to ACM Conference Proceedings WHO DO NOT WISH TO
% STRICTLY ADHERE TO THE SIGS (PUBS-BOARD-ENDORSED) STYLE.
% The 'sig-alternate.cls' file will produce a similar-looking,
% albeit, 'tighter' paper resulting in, invariably, fewer pages.
%
% ----------------------------------------------------------------------------------------------------------------
% This .tex file (and associated .cls V2.5) produces:
%       1) The Permission Statement
%       2) The Conference (location) Info information
%       3) The Copyright Line with ACM data
%       4) NO page numbers
%
% as against the acm_proc_article-sp.cls file which
% DOES NOT produce 1) thru' 3) above.
%
% Using 'sig-alternate.cls' you have control, however, from within
% the source .tex file, over both the CopyrightYear
% (defaulted to 200X) and the ACM Copyright Data
% (defaulted to X-XXXXX-XX-X/XX/XX).
% e.g.
% \CopyrightYear{2007} will cause 2007 to appear in the copyright line.
% \crdata{0-12345-67-8/90/12} will cause 0-12345-67-8/90/12 to appear in the copyright line.
%
% ---------------------------------------------------------------------------------------------------------------
% This .tex source is an example which *does* use
% the .bib file (from which the .bbl file % is produced).
% REMEMBER HOWEVER: After having produced the .bbl file,
% and prior to final submission, you *NEED* to 'insert'
% your .bbl file into your source .tex file so as to provide
% ONE 'self-contained' source file.
%
% ================= IF YOU HAVE QUESTIONS =======================
% Questions regarding the SIGS styles, SIGS policies and
% procedures, Conferences etc. should be sent to
% Adrienne Griscti (griscti@acm.org)
%
% Technical questions _only_ to
% Gerald Murray (murray@hq.acm.org)
% ===============================================================
%
% For tracking purposes - this is V2.0 - May 2012

\documentclass{sig-alternate-05-2015}
% Non-unicode must die
\usepackage[utf8]{inputenc}
\usepackage{alltt}
\usepackage{float}
\usepackage{listings}

\begin{document}

\lstset{language=C,breaklines=true}

\lstset{
  literate={í}{{\'i}}1
           {ç}{{\c c}}1
		   {¬}{{$\neg$}}1
		   {ã}{{\~a}}1
		   {ê}{{\^e}}1
}

\hyphenation{pri-mei-ros in-clu-í-ram in-dis-po-ní-vel ne-ces-sá-ria}

% Copyright
%\setcopyright{acmcopyright}
%\setcopyright{acmlicensed}
\setcopyright{rightsretained}
%\setcopyright{usgov}
%\setcopyright{usgovmixed}
%\setcopyright{cagov}
%\setcopyright{cagovmixed}


% DOI
%\doi{}

% ISBN
%\isbn{}

%Conference
%\conferenceinfo{PLDI '13}{June 16--19, 2013, Seattle, WA, USA}

%\acmPrice{\$15.00}

%
% --- Author Metadata here ---
%\conferenceinfo{WOODSTOCK}{'97 El Paso, Texas USA}
%\CopyrightYear{2007} % Allows default copyright year (20XX) to be over-ridden - IF NEED BE.
%\crdata{0-12345-67-8/90/01}  % Allows default copyright data (0-89791-88-6/97/05) to be over-ridden - IF NEED BE.
% --- End of Author Metadata ---

\newcommand{\banner}{\textit{banner} }

\floatstyle{ruled}
\newfloat{log}{thp}{lop}
\floatname{log}{Log}

\newfloat{program}{thp}{lop}
\floatname{program}{Programa}

\title{Banner Grabber em C utilizando conexões TCP}
%\subtitle{[Extended Abstract]
%\titlenote{A full version of this paper is available as
%\textit{Author's Guide to Preparing ACM SIG Proceedings Using
%\LaTeX$2_\epsilon$\ and BibTeX} at
%\texttt{www.acm.org/eaddress.htm}}}
%
% You need the command \numberofauthors to handle the 'placement
% and alignment' of the authors beneath the title.
%
% For aesthetic reasons, we recommend 'three authors at a time'
% i.e. three 'name/affiliation blocks' be placed beneath the title.
%
% NOTE: You are NOT restricted in how many 'rows' of
% "name/affiliations" may appear. We just ask that you restrict
% the number of 'columns' to three.
%
% Because of the available 'opening page real-estate'
% we ask you to refrain from putting more than six authors
% (two rows with three columns) beneath the article title.
% More than six makes the first-page appear very cluttered indeed.
%
% Use the \alignauthor commands to handle the names
% and affiliations for an 'aesthetic maximum' of six authors.
% Add names, affiliations, addresses for
% the seventh etc. author(s) as the argument for the
% \additionalauthors command.
% These 'additional authors' will be output/set for you
% without further effort on your part as the last section in
% the body of your article BEFORE References or any Appendices.

\numberofauthors{2} %  in this sample file, there are a *total*
% of EIGHT authors. SIX appear on the 'first-page' (for formatting
% reasons) and the remaining two appear in the \additionalauthors section.
%
\author{
% You can go ahead and credit any number of authors here,
% e.g. one 'row of three' or two rows (consisting of one row of three
% and a second row of one, two or three).
%
% The command \alignauthor (no curly braces needed) should
% precede each author name, affiliation/snail-mail address and
% e-mail address. Additionally, tag each line of
% affiliation/address with \affaddr, and tag the
% e-mail address with \email.
%
% 1st. author
\alignauthor
Renan Domingos Merlin Greca\\
			\affaddr{Universidade Federal do Paraná}\\
		    \email{renangreca@gmail.com}
% 2nd. author
\alignauthor
José Robyson Aggio Molinari\\
			\affaddr{Universidade Federal do Paraná}\\
			\email{aggio13@hotmail.com}
}
\date{5 September 2016}
% Just remember to make sure that the TOTAL number of authors
% is the number that will appear on the first page PLUS the
% number that will appear in the \additionalauthors section.

\maketitle

\section{Introdução}
Este relatório refere-se ao trabalho 1 da disciplina de Introdução a Segurança Computacional, cursada na Universidade Federal do Paraná no segundo semestre de 2016.
O objetivo do trabalho é criar um programa em C que estabelece conexões com uma determinada sequência de endereços IP e portas para receber o \banner dos serviços rodando nessas portas.
Chamamos esse procedimento de \textit{banner grabbing}.
Tal operação é interessante para agentes maliciosos buscando possíveis vulnerabilidades em servidores remotos, pois eles saberão com precisão quais IPs e portas podem apresentar um ponto de entrada.

Os desafios para o desenvolvimento deste projeto incluíram: (1) a interpretação da entrada; (2) o estabelecimento de uma conexão TCP e o recebimento de um \textit{banner}; (3) a iteração sobre toda a sequência de IPs/portas; e (4) a otimização da execução ao evitar conexões inválidas.

\subsection{Banners}

Ao estabelecer uma conexão com um servidor remoto em determinada porta, os primeiros bytes recebidos podem gerar uma \textit{string} chamada \textit{banner}.
O \banner contém detalhes (por exemplo, a versão) do serviço disponível na porta.
No entanto, nem todo serviço possui um banner, então, mesmo que exista algo disponível em uma porta, pode ser que uma \textit{string} vazia seja encontrada.

Suponha um servidor qualquer rodando um serviço de SSH na porta 22 (a porta padrão do SSH).
Um exemplo de \banner que pode ser encontrado nessa porta é:
\begin{verbatim}
	SSH-1.99-OpenSSH_2.9p2
\end{verbatim}

No caso, o \banner acima indica que o servidor está rodando o serviço OpenSSH, versão 2.9, que utiliza a versão 1.99 do protocolo SSH.
Com essas informações em mão, um atacante pode buscar, com relativa facilidade, vulnerabilidades da versão 2.9 do OpenSSH ou da versão 1.99 do SSH.

\section{Desenvolvimento}
Conforme mencionado na introdução, o desenvolvimento do projeto contou com quatro desafios principais. Nesta seção, esses desafios são detalhados e suas soluções são explicadas.

\subsection{Entrada}
O primeiro desafio, e o menos relevante para o conceito do projeto, foi interpretar a entrada do programa. O programa desenvolvido chama-se \texttt{recon} e aceita entrada no seguinte formato:
\begin{verbatim}
	recon <ip> [port]
\end{verbatim}
Tanto \texttt{ip} quanto \texttt{port} podem ser um valor único ou uma faixa separada por \texttt{-}.
Adicionalmente, o parâmetro \texttt{port} é opcional; a ausência dele causará o uso de todas as portas, ou seja, de 1 a 65535.
Portanto, alguns exemplos de chamadas do programa são:
\begin{verbatim}
	recon 192.168.1.1 1-100
	recon 192.168.1.1-10
	recon 192.168.1.1-10 1-100
\end{verbatim}
Dadas entradas nesses formatos, foi necessário utilizar as funções \texttt{strsep} e \texttt{strcat}, presentes na biblioteca \texttt{<string.h>}, para separar e concatenar \textit{strings}.
Adicionalmente, foi necessária a função \texttt{atoi} da biblioteca padrão, \texttt{<stdlib.h>}.

\begin{program}
\begin{lstlisting}
while ((token = strsep(&ipstring, "-")) != NULL) {
    iprange[i] = token;
    i++;
}
while (((token = strsep(&iprange[0], ".")) != NULL) && i<3 ) {        
    strcat(subnet, token);
    strcat(subnet, ".");
    i++;
}
int ip_min = atoi(token);
int ip_max;
if (iprange[1] != NULL) {
    ip_max = atoi(iprange[1]);
}
\end{lstlisting}
\caption{Segmento do programa que separa o endereço IP; uma operação similar é realizada para as portas.}
\end{program}


O Programa 1 mostra os laços que utilizam a função \texttt{strsep} para separar a primeira parte da entrada, a faixa de endereços IP, usando os caracteres \texttt{-} e \texttt{.}.
A subrede do endereço é formada ao concatenar apenas as três primeiras partes do endereço IP; o quarto octeto será adicionado apenas durante a iteração sobre a faixa de IPs.

\subsection{Conexão TCP}
Para estabelecer a conexão TCP, primeiramente foi criado um programa separado que recebe apenas um endereço IP e uma porta e adquire o \textit{banner}.
Tutoriais para conexão TCP em C podem ser facilmente encontrados na Internet, então simplesmente foi utilizado um deles como base para o programa.
O programa abre um \textit{socket} utilizando os protocolos TCP e IP, estabelece uma conexão (por meio da função \texttt{connect}) com o servidor remoto usando o conjunto de endereço IP e porta.
Por fim, ele utiliza a função \texttt{read} para obter uma sequência de 255 bytes, que, ao ser interpretado como uma \textit{string}, é o \textit{banner}.

Algumas bibliotecas do C são necessárias para sua execução: \texttt{<sys/types.h>}, \texttt{<sys/socket.h>}, \texttt{<netinet/in.h>}, \texttt{<netdb.h>} e \texttt{<unistd.h>}.

O Programa 2 mostra as principais operações para estabelecer essa conexão.
No código completo há tratamento de erros e várias operações secundárias necessárias.
A constante \texttt{BUFFER\_SIZE} é 256.

\begin{program}
\begin{lstlisting}
// Abre o socket
sockfd = socket(AF_INET, SOCK_STREAM, 0);
// Monta as structs nos formatos corretos para a função connect
server = gethostbyname(ip);
...
serv_addr.sin_port = htons(port);
// Tenta estabelecer conexão
connect(sockfd,(struct sockaddr *)&serv_addr,sizeof(serv_addr));
// Lê os primeiros bytes recebidos
n = read(sockfd,buffer,BUFFER_SIZE-1);
\end{lstlisting}
\caption{As operações principais, fora de contexto, para receber os dados com TCP.}
\end{program}

\subsection{Iteração sobre vários endereços e portas}
\begin{program}
\begin{lstlisting}
for (int ipaddr=ip_min; ipaddr<=ip_max; ipaddr++) {
	for (int port=port_min; port<=port_max; port++) {
		/* conexão e banner grab */
		close(sockfd);
	}
}
\end{lstlisting}
\caption{Laços para a iteração sobre todos os IPs e portas, mais a função de fechamento de socket que mostrou-se necessária.}
\end{program}

A princípio, a iteração sobre diversos endereços IP e suas portas é trivial, quando consideradas as soluções dos desafios acima.
Com as faixas de IPs e portas em mãos, basta utilizar dois laços aninhados e executar a conexão para cada combinação possível.
No entanto, ao implementar essa solução, foi descoberto que as conexões deixavam de ocorrer após exatamente 254 conexões, ou tentativas, corretas.

A causa do problema era que, no programa que estabelecia uma conexão TCP isolada, um \textit{socket} era aberto mas não era fechado, e essa falha foi replicada ao adaptar o programa para a forma iterativa.
Portanto, foi necessário adicionar a função \texttt{close} usando o descritor do \textit{socket} como parâmetro para fechá-lo após cada iteração. Desta forma, foi possível testar centenas de milhares de portas sequencialmente sem problemas.

Os laços são mostrados no Programa 3, mostrando a finalidade das variáveis \texttt{ip\_min} e \texttt{ip\_max} declaradas no Programa 1 e suas equivalentes para a faixa de portas.

\subsection{Otimização das conexões}
Durante os testes, foi observado que, sob determinadas circunstâncias, uma conexão era estabelecida com o servidor, mas passava alguns segundos tentando obter um \textit{banner} sem sucesso.
Aguardar alguns segundos é aceitável ao verificar uma sequência curta de portas, mas é inviável em uma sequência de milhares.
Portanto, foi utilizada a biblioteca \texttt{<poll.h>}, que permite o uso de \textit{timeout} para que, se a obtenção do \banner demorar mais do que 0,1 segundo, assume-se que o \banner é vazio e o programa pode continuar para a próxima porta.

O Programa 4 mostra como foi utilizada a biblioteca \texttt{poll}.
A função \texttt{read}, vista no Programa 2, encontra-se no caso \textit{default}.

\begin{program}
\begin{lstlisting}
fd.fd = sockfd;
fd.events = POLLIN;
ret = poll(&fd, 1, 100);
      /* 0.1 second timeout */
switch (ret) {
    case -1:
        // Erro
        break;
    case 0:
        // Timeout 
        break;
    default:
        // Conexão; pegar dados
        break;
}
\end{lstlisting}
\caption{O \texttt{switch-case} acionado pela biblioteca \texttt{poll} que permite um \textit{timeout} nas conexões.}
\end{program}

Adicionalmente, quando algum dos endereços IP estava completamente inativo (ou seja, não havia servidor rodando com aquele endereço), o programa levava alguns segundos para tentar estabelecer uma conexão com cada uma das possíveis portas.
Assim como acima, isso era admissível.
Para evitar isso, foi utilizada a biblioteca \texttt{<errno.h>}, que permite uma identificação precisa do motivo pelo qual uma conexão falhou.

Foi descoberto que, quando um endereço IP está indisponível, a função \texttt{connect} falha com o erro \texttt{51}, então o restante das portas daquele IP podem ser ignoradas.
Outros erros geralmente indicam a indisponibilidade apenas da porta tentada.

\begin{program}
\begin{lstlisting}
if (connect(sockfd,(struct sockaddr *)&serv_addr,sizeof(serv_addr)) < 0) {
    close(sockfd);

    if (errno == 51) { 
        // Erro 51 indica endereço IP indisponível
        port = port_max;
    } // Outros erros podem ser porta inativa, timeout da conexão, etc.
    continue;   
}
\end{lstlisting}
\caption{Captação de erro de conexão; erro \texttt{51} indica que todo o endereço IP está inativo.}
\end{program}

\section{Resultados}
Mesmo com as otimizações acima, a execução do programa pode ser consideravelmente demorada.
Como solicitado no enunciado do projeto, o teste final foi feito usando a seguinte entrada:
\begin{verbatim}
	recon 200.238.144.20-29
\end{verbatim}
Ou seja, uma faixa de 10 endereços IP e todas as suas portas.

Como pode ser visto abaixo, o programa descobriu uma série de informações sobre os servidores rodando em IPs finalizados em 27, 28 e 29.
No entanto, como os endereços IPs finalizados com os números entre 20 e 26 estavam inativos, eles foram pulados após a verificação de apenas uma porta ao receber um erro \texttt{51}.


\begin{log}
\centering
\begin{lstlisting}
Varredura iniciada em Thu Sep  1 19:25:24 2016
IP: 200.238.144.20-29
Portas: (null)
---
200.238.144.27    21	 
200.238.144.27    22      SSH-1.99-OpenSSH_2.9p2
200.238.144.27    111     
200.238.144.27    32768     
200.238.144.28    22      SSH-2.0-OpenSSH_6.6.1p1 Ubuntu-2ubuntu2.6
200.238.144.28    5001    ’
200.238.144.28    8080     
200.238.144.29    21      220 Welcome to the ftp service
200.238.144.29    22      SSH-2.0-OpenSSH_5.9p1 Debian-5ubuntu1.9
200.238.144.29    42     
200.238.144.29    80     
200.238.144.29    443     
200.238.144.29    1433     
200.238.144.29    3306    4
200.238.144.29    5060     
200.238.144.29    5061     
\end{lstlisting}
\caption{log do programa}
\end{log}

Como pode ser visto, os serviços de FTP e SSH são os mais fáceis de detectar.
Outros serviços, como aqueles rodando na porta 5001 do IP 28 e na porta 3306 do IP 29, podem ter oferecido bytes que deveriam ser interpretados de forma diferente, portanto não fazem sentido quando vistos como uma \textit{string}.

Os serviços restantes têm \textit{banners} vazios.
Isso pode ocorrer porque o serviço instanciado naquela porta utiliza um protocolo (como, por exemplo, SMTP) que utiliza pacotes mais elaborados, portanto o \banner pode não estar nos primeiros bytes.
Outra possibilidade é que seja um serviço de HTTP, que retorna pacotes apenas após receber alguma solicitação e, portanto, ocorre um \textit{timeout} na conexão.

De qualquer maneira, essa execução pode resumir as centenas de milhares de portas inicialmente recebidas para apenas dezesseis.
Com essa lista de IPs e portas em mão, é possível construir um algoritmo que tente efetuar conexões de outros tipos com cada uma delas, ou mesmo tentar acessá-las manualmente.

%
% The following two commands are all you need in the
% initial runs of your .tex file to
% produce the bibliography for the citations in your paper.
\bibliographystyle{abbrv}
\bibliography{sigproc}  % sigproc.bib is the name of the Bibliography in this case
% You must have a proper ".bib" file
%  and remember to run:
% latex bibtex latex latex
% to resolve all references
%
% ACM needs 'a single self-contained file'!
%
%APPENDICES are optional
%\balancecolumns
%\balancecolumns % GM June 2007
% That's all folks!
\end{document}
